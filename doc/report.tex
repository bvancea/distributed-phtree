\documentclass[11pt,a4paper]{globis-book}

\usepackage{graphicx}
\usepackage[helvetica]{quotchap}
\usepackage{times}
\usepackage{ethfont}
\usepackage[british]{babel}
\usepackage{longtable}
\usepackage{fancyhdr}
\usepackage{shadethm}
\usepackage{makeidx}
\usepackage[a4paper,portrait,twoside,inner=3.25cm,outer=3.5cm,top=3.5cm,bottom=4.0cm]{geometry}
\usepackage{hyperref}
\usepackage{globis}

\renewcommand{\sectfont}{\sffamily\bfseries\Huge}

\setlength{\shadedtextwidth}{\textwidth}
\setlength{\shadeleftshift}{3mm}
\setlength{\shaderightshift}{3mm}
\addtolength{\shadedtextwidth}{-\shadeleftshift}
\addtolength{\shadedtextwidth}{-\shaderightshift}
\setlength{\parindent}{0pt}
\setlength{\parskip}{5pt}

\sloppy

\pagestyle{fancy}
\fancyhf{}
\fancyhead[LE,RO]{\sffamily\bfseries\small\thepage}
\fancyhead[LO]{\sffamily\bfseries\small\leftmark}
    \fancyhead[RE]{\sffamily\bfseries\small\rightmark}
\renewcommand{\headrulewidth}{0.1pt}
\renewcommand{\footrulewidth}{0pt}

\fancypagestyle{plain}{
    \fancyhf{}
    \fancyfoot[C]{\sffamily\bfseries\small\thepage}
    \renewcommand{\headrulewidth}{0pt}
    \renewcommand{\footrulewidth}{0pt}
}

% Please adapt the following fields if necessary!
\hypersetup{
    pdftitle = Distributed Ph Tree, 
    pdfauthor = Bogdan Vancea,
    pdfsubject = Master Thesis,
    hidelinks,
    plainpages = false,
    bookmarksnumbered = true
} 

\raggedbottom

% Please adapt the following fields if necessary!
\title{Cluster-Computing and Parallelisation for the
    Multi-Dimensional PH-Index}
\category{Master Thesis} 
\author{Bogdan Aurel Vancea}
\email{$<$bvancea@student.ethz.ch$>$}
\professor{Prof. Dr. Moira C. Norrie}
\assistant{Tilmann Zaeschke \\Christoph Zimmerli}
\group{Global Information Systems Group}
\institute{Institute of Information Systems}
\department{Department of Computer Science}
\school{ETH Zurich}
\version{}
\date{\today}
\copyrightyear{2014}

\makeindex

\begin{document}

\frontmatter
\maketitlepage
\cleardoublepage
\pdfbookmark{Contents}{toc}

\chapter*{Abstract}

Here comes the abstract.

\tableofcontents

\mainmatter

% Here comes the content

\chapter{Introduction}
\section{Motivation}
\textit{Meta - will be removed after editing.}
\textit{This subsection will describe the context in which this work is placed and why this work is needed.}

\section{Objectives}

\textit{Meta - will be removed after editing.}
\textit{Describe the objectives of this thesis: the distribution of the index and the addition of the concurrency support}
\section{Thesis outline}

\textit{Meta - will be removed after editing.}
\textit{Give an overview of what each chapter will contain}
\label{sec:title}

\chapter{Background}
\textit{Meta - will be removed after editing.}
\textit{This section should provide an overview of the PhTree. It should also present important previous work concerning distributed indexes and parallel data structures.}

This is an example of how to cite a scientific publication~\cite{murolo2013} from your bibliography (BibTeX\footnote{\url{http://en.wikipedia.org/wiki/BibTeX}} file). And this example shows how you create links within your documents, e.g. link to section~\ref{sec:title}.

\section{The PhTree}
\textit{Meta - will be removed after editing.}
\textit{Provide on overview of the PhTree. Should not go into too many details here, refer to PhTree paper.}

\section{Related work}

\textit{Meta - will be removed after editing.}
\textit{Present the relevant related work.}
\subsection{Distributed Indexes}
\subsection{Concurrent data structures}

\chapter{Index distribution}

\textit{Meta - will be removed after editing.}
\textit{This chapter should focus on how the distributed index was implemented: how the data was split across the cluster nodes, the manner in which the queries are executed and how the entry load balancing is performed.} 
\section{Challenges}

\textit{Meta - will be removed after editing.}
\textit{Present the callenges of implementing a distributed system : scalability, load balancing, etc. Should not focus on issues like security, availability as those are not relevant to this report.}
\section{Distibution strategies}

\textit{Meta - will be removed after editing.}
\textit{Present the possible ways in which the entries can be distributed across the cluster nodes. Talk about the advantages and disaadvantages of each approach. Say which approach was chosen and why.}
\subsection{Hashing}
\subsection{Spatial splitting}
\subsection{Z-Order curve splitting}

\section{Algorithms}

\textit{Meta - will be removed after editing.}
\textit{This section should explain how the queries should be executed on the distributed system. Present the load balancing algorithm}.
\subsection{Point queries}
\subsection{Range queries}
\subsection{Nearest neighbour queries}
\subsection{Entry load balancing}

\section{Architecture}

\textit{Meta - will be removed after editing.}
\textit{This section should explain how the queries should be executed on the distributed system. Present the load balancing algorithm}.

\chapter{Concurrency}

\textit{Meta - will be removed after editing.}
\textit{Present the concurrency strategies that could be added to the PhTree and explain the consistency model associated with each strategy.}

The PhTree does not currently support concurrent write operations. There are several strategies that could be employed to add concurrent writes.
\section{Challenges}
\section{Concurrency strategies}
\subsection{Copy-on-Write}
\subsection{Locking}

\chapter{Implementation}
\textit{Meta - will be removed after editing.}
\textit{Present the implementation architecture and the techologies used.}
\section{System description}
\textit{Meta - will be removed after editing.}
\textit{Describe the system, include class/deployment diagrams.}
\section{Technologies}
\textit{Meta - will be removed after editing.}
\textit{Describe the technologies used, the reasons for which these technologies were chosen and any alternatives.}

\chapter{Evaluation}
\textit{Meta - will be removed after editing.}
\textit{Explain how the system should be evaluated, present and explain the benchmarks}
\chapter{Conclussions}

\textit{Meta - will be removed after editing.}
\textit{Conclude the report. This should reiterate the main points of the report and try to mirror the introduction}
\chapter{Future work}

\textit{Meta - will be removed after editing.}
\textit{Present the points that were not tackled by the thesis and talk about possible future work.}

\appendix

\listoffigures
\listoftables

\chapter*{Acknowledgements}

\newpage
\thispagestyle{empty}

\bibliographystyle{plain}
\bibliography{bibliography}

\end{document} 
